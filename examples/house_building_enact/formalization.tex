\documentclass{article}

\usepackage{amsthm}
\usepackage{amsmath}

\newtheorem{definition}{Definition}

\begin{document}


\section{Prelimimary definitions}

\begin{definition}[Enact Rule]
    An enact rule is a pair $(y,x)$ where (i)~$y$ is a logical formula and $x$ is an expected enactment effect, expressed as an identifier followed by $n\geq 0$ arguments.
\end{definition}

\begin{definition}[Enact Specificaton]
    An enact specification $\mathcal{E}$ is a set of enact rules.
\end{definition}

\begin{definition}[Institutional state]
  The institutional state $I$ is a set of facts representing properties of all the institutional components of the system (constitutive rules, norms, sanctions, etc.).  
\end{definition}
It is assumed that 
(i)~the term $y$ of an enact rule $(x,y)$ is evaluated with respect to a representation of $I$; 
(ii)~this representation is allways consistent;
(iii)~if ~$i\in I$, then $\{i\}^\prime\cup I=I\rightarrow i^\prime=i$.

\section{Enactment dynamics}


\begin{align}
\frac{\exists_{(y,x)\in\mathcal{E}}I\models y\theta \ \ \ x\theta\notin E}
{E\longrightarrow E\cup x\theta}
\end{align}
Informally, if a formula $y$, under a substitution $\theta$, is true with respect the current institutional state, 
then the current enact effects include the  $x$, under $\theta$.


\begin{align}
  \frac{\exists_{(y,x)\in\mathcal{E}}x\theta\in E\wedge I\not\models y\theta}
  {E\longrightarrow E\setminus x\theta}
\end{align}
\end{document}

